%% The following is a directive for TeXShop to indicate the main file
%%!TEX root = diss.tex

\chapter{Preface}

Parts of this dissertation have been previously published with various co-authors:

A version of {\bf \autoref{ch:typology}} has been published as {\it A Multi-Level Typology of Abstract Visualization Tasks} by {\bf Matthew Brehmer} and Tamara Munzner; in IEEE Transactions on Visualization and Computer Graphics (Proceedings of InfoVis 2013), 19(12), p. 2376--2385~\cite{Brehmer2013}\footnote{\url{http://dx.doi.org/10.1109/TVCG.2013.124}}.
I conducted the literature review. 
Tamara and I both contributed to the meta-analysis of the literature and writing.
A modified version of the task typology proposed in this chapter appears in {\it Visualization Analysis and Design} by Tamara Munzner (AK Peters Visualization Series, CRC Press, 2014)~\cite{Munzner2014}.

A version of {\bf \autoref{ch:drvistasks}} has been published as {\it Visualizing Dimensionally Reduced Data: Interviews with Analysts and a Characterization of Task Sequences} by {\bf Matthew Brehmer}, Michael Sedlmair, Stephen Ingram, and Tamara Munzner; in Proceedings of the ACM Workshop on Beyond Time and Errors: Novel Evaluation Methods For Information Visualization (BELIV 2014), p.1-8~\cite{Brehmer2014b}\footnote{\url{http://dx.doi.org/10.1145/2669557.2669559}}. 
This publication was preceded by a technical report entitled {\it Dimensionality Reduction in the Wild: Gaps and Guidance} by Michael Sedlmair, {\bf Matthew Brehmer}, Stephen Ingram and Tamara Munzner (UBC CS TR-2012-03)~\cite{Sedlmair2012b}\footnote{\url{http://cs.ubc.ca/cgi-bin/tr/2012/TR-2012-03}}, and by an unpublished manuscript entitled {\it Dimensionality Reduction in the Wild} by Michael Sedlmair, {\bf Matthew Brehmer}, Stephen Ingram and Tamara Munzner (2013)\footnote{Included in the appendices as \autoref{app:drvistasks:dritw}.}. 
Michael conducted the majority of the interviews with analysts between 2010 and 2012. 
All authors contributed to the initial analysis of the collected data. 
For the BELIV paper~\cite{Brehmer2014b}, I re-analyzed this data using the task typology described in \autoref{ch:typology}. 
I performed the majority of the writing for the BELIV 2014 submission (which omitted much of the material from the earlier technical report and manuscript); Tamara and Michael contributed to the editing process.

A version of {\bf \autoref{ch:overview}} has been published as {\it Overview: The Design, Adoption, and Analysis of a Visual Document Mining Tool For Investigative Journalists} by {\bf Matthew Brehmer}, Stephen Ingram, Jonathan Stray, and Tamara Munzner; in IEEE Transactions on Visualization and Computer Graphics (Proceedings of InfoVis 2014), 20(12), p. 2271--2280~\cite{Brehmer2014}\footnote{\url{http://dx.doi.org/10.1109/TVCG.2014.2346431}}. 
{\it Overview} was developed by Jonathan Stray with contributions from Jonas Karlsson, Adam Hooper, and Stephen Ingram. 
Stephen's algorithmic contributions are documented in greater detail in an earlier technical report~\cite{Ingram2012} and in his PhD dissertation~\cite{Ingram2013}.
I conducted a post-deployment evaluation of {\it Overview} and its use by investigative journalists. Jonathan and I interviewed the {\sc tulsa}, {\sc ryan}, and {\sc dallas} journalists; I interviewed the {\sc guns} journalist, while Jonathan interviewed the {\sc newyork} journalist. 
Jonathan conducted the think-aloud evaluation with journalists.
I performed the analysis of the interview data (including transcripts and screen captures), as well as the {\it Overview} log data.
I performed the majority of the writing for the InfoVis 2014 submission, while Tamara and Jonathan contributed to the editing process.

A version of {\bf \autoref{ch:emu}} has been published as {\it Matches, Mismatches, and Methods: Multiple-View Workflows for Energy Portfolio Analysis} by {\bf Matthew Brehmer}, Jocelyn Ng, Kevin Tate, and Tamara Munzner; in IEEE Transactions on Visualization and Computer Graphics (Proceedings of InfoVis 2015), 22(1), p. 449--458~\cite{Brehmer2015}\footnote{\url{http://dx.doi.org/10.1109/TVCG.2015.2466971}}.
I conducted the work domain analysis, sandbox prototyping, and the analysis of feedback on prototype designs from energy analysts. 
Jocelyn and I both contributed to the workflow design.
Kevin initiated the project and provided feedback on my process during my internship at EnerNOC (then Pulse Energy); he also provided introductions to energy analysts.
EnerNOC's Energy Manager development team, led by Cailie Crane and Reetu Mutti, implemented some of our prototype designs into a new commercial version of Energy Manager.
I performed the majority of the writing for the InfoVis 2015 submission, while Tamara and Jocelyn contributed to the editing process.

All images in \autoref{ch:typology}, \autoref{ch:overview}, and \autoref{ch:emu} are reprinted with the permission of the IEEE.
% , with the exception of 
% \autoref{typology:fig:chropleth}, which was produced by Wikimedia Commons contributor Deepthiyathiender is licensed under the Creative Commons Attribution 4.0 International license (\ccLogo~BY 4.0), and 
\autoref{overview:fig:warlogs} is a detail from \autoref{app:overview:fig:warlogs}, an image produced by Jonathan \citet{Stray2010}.
All images in \autoref{ch:drvistasks} are reprinted with the permission of the ACM, with the exception of \autoref{drvistasks:fig:drviztasks-name-dims-tenenbaum}, which appears in \citet{Tenenbaum2000} and is reprinted with the permission of the AAAS.
This dissertation includes several illustrations that were originally created by Eamonn Maguire for {\it Visualization Analysis and Design} by Tamara Munzner~\cite{Munzner2014}, including \autoref{fig:nested-model}, \autoref{fig:typology-extension}, \autoref{fig:typology-actions}, \autoref{fig:typology-targets}, \autoref{fig:typology-vad-what}, and \autoref{fig:typology-vad-how}; these illustrations are available for use under the Creative Commons Attribution 4.0 International license (\ccLogo~BY 4.0).

The studies described in this dissertation were conducted with the approval of the UBC Behavioural Research Ethics Board (BREB): certificate number H10-03336.
