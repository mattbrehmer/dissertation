% This file provides examples of some useful macros for typesetting
% dissertations.  None of the macros defined here are necessary beyond
% for the template documentation, so feel free to change, remove, and add
% your own definitions.
%
% We recommend that you define macros to separate the semantics
% of the things you write from how they are presented.  For example,
% you'll see definitions below for a macro \file{}: by using
% \file{} consistently in the text, we can change how filenames
% are typeset simply by changing the definition of \file{} in
% this file.
% 
%% The following is a directive for TeXShop to indicate the main file
%%!TEX root = diss.tex

\definecolor{darkGreen}{RGB}{0, 100, 0}
\definecolor{nmGreen}{RGB}{225, 255, 194}
\definecolor{nmOrange}{RGB}{254, 225, 194}
\definecolor{nmYellow}{RGB}{255, 255, 208}
\definecolor{indigo}{RGB}{75, 0, 130}

\newcommand*\rot{\rotatebox{90}}
\newcommand*\OK{\ding{52}}

\newcommand*\rotsmall{\scriptsize \rotatebox{90}}
\newcommand*\OKemu{\ding{51}}

\newcommand*\match{\textcolor{darkGreen}{\ding{52}}}
\newcommand*\mismatch{\textcolor{red}{\ding{54}}}
\newcommand*\posmatch{\textcolor{blue}{{\bf ?}}}

\newcommand{\bstart}[1]{\vspace{1mm} \noindent{\textbf{#1:}}}

\newcommand{\bqstart}[1]{\vspace{1mm} \noindent{\textbf{#1}}}

\newcommand{\bscstart}[1]{\vspace{1mm} \noindent{\sc{\textbf{#1:}}}}

\newcommand{\NA}{\textsc{n/a}}	% for "not applicable"

% author comments
\definecolor{matt}{RGB}{27,158,119}
\definecolor{tamara}{RGB}{117,112,179}
\definecolor{joanna}{RGB}{217,95,2}
\definecolor{ron}{RGB}{102,166,30}

\newcommand{\mb}[1]{\textcolor{matt}{\textbf{MB: #1}}}
\newcommand{\tm}[1]{\textcolor{tamara}{\textbf{TM: #1}}}
\newcommand{\jm}[1]{\textcolor{joanna}{\textbf{JM: #1}}}
\newcommand{\rr}[1]{\textcolor{ron}{\textbf{RR: #1}}}
% \newcommand{\mb}[1]{}
% \newcommand{\tm}[1]{}
% \newcommand{\jm}[1]{}
% \newcommand{\rr}[1]{}

\newcommand{\etal}{et al.}
\newcommand{\eg}{e.g., }
\newcommand{\ie}{i.e., }
\newcommand{\cf}{c.f. }

% Some useful macros for typesetting terms.
\newcommand{\file}[1]{\texttt{#1}}
\newcommand{\class}[1]{\texttt{#1}}
\newcommand{\latexpackage}[1]{\href{http://www.ctan.org/macros/latex/contrib/#1}{\texttt{#1}}}
\newcommand{\latexmiscpackage}[1]{\href{http://www.ctan.org/macros/latex/contrib/misc/#1.sty}{\texttt{#1}}}
\newcommand{\env}[1]{\texttt{#1}}
\newcommand{\BibTeX}{Bib\TeX}

\newcounter{papernumber} 
\setcounter{papernumber}{0}
\renewcommand{\thepapernumber}{~[R-\arabic{papernumber}]}

\newcounter{rownumber}[figure] 
\setcounter{rownumber}{1}
\renewcommand{\therownumber}{A\arabic{rownumber}}

\newcounter{prownumber}[figure] 
\setcounter{prownumber}{1}
\renewcommand{\theprownumber}{\Roman{prownumber}}

% Define a command \doi{} to typeset a digital object identifier (DOI).
% Note: if the following definition raise an error, then you likely
% have an ancient version of url.sty.  Either find a more recent version
% (3.1 or later work fine) and simply copy it into this directory,  or
% comment out the following two lines and uncomment the third.
\DeclareUrlCommand\DOI{}
\newcommand{\doi}[1]{\href{http://dx.doi.org/#1}{\DOI{doi:#1}}}
%\newcommand{\doi}[1]{\href{http://dx.doi.org/#1}{doi:#1}}

% Useful macro to reference an online document with a hyperlink
% as well with the URL explicitly listed in a footnote
% #1: the URL
% #2: the anchoring text
\newcommand{\webref}[2]{\href{#1}{#2}\footnote{\url{#1}}}

% epigraph is a nice environment for typesetting quotations
\makeatletter
\newenvironment{epigraph}{%
	\begin{flushright}
	\begin{minipage}{\columnwidth-0.75in}
	\begin{flushright}
	\@ifundefined{singlespacing}{}{\singlespacing}%
    }{
	\end{flushright}
	\end{minipage}
	\end{flushright}}
\makeatother

% \FIXME{} is a useful macro for noting things needing to be changed.
% The following definition will also output a warning to the console
\newcommand{\FIXME}[1]{\typeout{**FIXME** #1}\textbf{[FIXME: #1]}}

% END
