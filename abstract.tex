%% The following is a directive for TeXShop to indicate the main file
%%!TEX root = diss.tex

\chapter{Abstract}

Why do people visualize data?
% Why do people use computers to make pictures of numbers, words, places, and other things?

People visualize data either to consume or produce information relevant to a domain-specific problem or interest.
Visualization design and evaluation involves a mapping between domain problems or interests and appropriate visual encoding and interaction design choices.
This mapping translates a domain-specific situation into \textsl{abstract visualization tasks}, which allows for succinct descriptions of tasks and task sequences in terms of {\it why} data is visualized, {\it what} dependencies a task might have in terms of {\it input} and {\it output}, and {\it how} the task is supported in terms of visual encoding and interaction design choices.
Describing tasks in this way facilitates the comparison and cross-pollination of visualization design choices across application domains; the mapping also applies in reverse, whenever visualization researchers aim to contextualize novel visualization techniques.
% People make these types of pictures in order to understand interesting stuff that happens in our world or to share what they understand with other people. In order to make these types of pictures and to make sure that they are good, you have to understand both the problems or interests that people have and all of the types of pictures that might help them. So you need to figure out what people are going to do with these pictures, and you need to find the right words to explain what people can do and in which order they are going to do them; this will help you to pick the best types of pictures to show these people, and to pick the ways that people can play with these pictures using a computer.
% Once you have found this right words to explain what people do with these pictures, you might find that the way you explained it and the pictures you picked could also help other people with different problems or interests. The way you explained what people do might also help other people who make pictures but don’t yet know what they’re good for.

In this dissertation, we present multiple instances of visualization task abstraction, each integrating our proposed typology of abstract visualization tasks.
We apply this typology as an analysis tool in an interview study of individuals who visualize dimensionally reduced data in different application domains, in a post-deployment field study evaluation of a visual analysis tool in the domain of investigative journalism, and in a visualization design study in the domain of energy management.
% In this book, I talk about a few times when we studied people who use these types of pictures as part of their job, and each time we used a set of really good words to explain what they do with these pictures. One time, we spoke to people who had very different jobs but they all used a computer to make pictures out of lots and lots of numbers. Another time, we spoke to people who write news stories, and they too made pictures using a computer, but this time they made pictures out of lots and lots of words. And then there was another time when we helped people who have the job of looking at how much power a building uses and deciding if the building needs to be changed in order to save power; we spoke to these people and found better pictures that will help them to do their job.

In the interview study, we draw upon and demonstrate the descriptive power of our typology to classify five task sequences relating to visualizing dimensionally reduced data. 
This classification is intended to inform the design of new tools and techniques for visualizing this form of data.
% During that time that we spoke to people who made pictures out of lots and lots of numbers, we used our set of really good words to explain that there are five reasons why these people made these pictures, pictures that were pretty simple: just little points in a computer window. Now that other people know about these reasons, they can make better pictures for showing lots and lots of numbers using computers.

In the field study, we draw upon and demonstrate the descriptive and evaluative power of our typology to evaluate {\it Overview}, a visualization tool for investigating large text document collections. 
After analyzing its adoption by investigative journalists, we characterize two abstract tasks relating to document mining and present seven lessons relating to the design of visualization tools for document data.
% During that time that we spoke to people who write news stories, we used our set of really good words again to explain why these people make pictures out of lots and lots of words. They turned all of these words into a tree in their computer window, and by looking at this tree, they found interesting things and wrote news stories about them. We showed that the thing that they used to make these pictures did a great job, and we learned seven things about making these types of pictures using a computer.

In the design study, we demonstrate the descriptive, evaluative, and generative power of our typology and identify matches and mismatches between visualization design choices and three abstract tasks relating to time series data. 
% Finally, during that time that we helped the power people, we used our set of really good words yet again. We learned that the power people do three things during their job, and they all have to do with understanding numbers that change over time, numbers that show how much power a building uses all day long. In the end, we found the best pictures that will help the power people do these three things, but we tried a lot of other pictures first. And because we used our set of really good words to explain what the power people do, we might help other people with different jobs but still need to make pictures that show changing numbers.

Finally, we reflect upon the impact of our task typology.
% At the end of this book, we talk about how our set of really good words has helped lots of people who make pictures using computers.

% Embed version information inline -- you should remove this from your
% % dissertation
% \vfill
% \begin{center}
% \begin{sf}
% \fbox{Version: 5 - \today
}
% \end{sf}
% \end{center}
